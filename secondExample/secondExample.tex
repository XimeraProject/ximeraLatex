\documentclass[handout,instructornotes]{ximera}
%% handout
%% space
%% newpage
%% numbers
%% instructornotes

%% You can put user macros here
%% However, you cannot make new environments

\graphicspath{{./}{firstExample/}{secondExample/}}

\usepackage{tikz}
\usepackage{tkz-euclide}
\usetkzobj{all}
\pgfplotsset{compat=1.7} % prevents compile error.

\tikzstyle geometryDiagrams=[ultra thick,color=blue!50!black]
 %% we can turn off include when making a master document

\outcome{Understand a second example of the Ximera style.}
\outcome{See how to include graphics.}

\title{Second example}

\begin{document}
\begin{abstract}
In this activity we give a second example.
\end{abstract} 
\maketitle
\begin{instructorNotes}
  Here we see a multi-part question.
\end{instructorNotes}

\begin{instructorIntro}
  This should tell me something
\end{instructorIntro}

Here we have a multi-part question with free-response.

\begin{question} 
Suppose you are standing on a bridge that is 60 meters above
sea-level. You toss a ball up into the air with an initial velocity of
30 meters per second.  If $t$ is the time (in seconds) after we toss
the ball, then the height at time $t$ is approximately $f(t) = -5 t^2
+30t+60$. What does $f(2)$ mean in our context?
\begin{hint}
We want an answer in the context of the problem. 
\end{hint}
\begin{freeResponse}
The value $f(2)$ is the height of the ball after $2$ seconds.
\end{freeResponse}
Now suppose $t$ is such that $f(t) = 100$. What does this mean in our
context?
\begin{hint}
We want an answer in the context of the problem. 
\end{hint}
\begin{freeResponse}
These value of $t$ are the times when the ball is at 100 meters above sea level.\end{freeResponse}
Finally, if $h$ is a small positive value what is the meaning of
$f(t+h)$? How does this compare to the meaning of $f(t)+h$?
\begin{hint}
We want an answer in the context of the problem. 
\end{hint}
\begin{freeResponse}
The value $f(t+h)$ gives the height of the ball slightly after time
$t$. On the other hand, the value $f(t)+h$ gives a height just higher
than the ball at time $t$.
\end{freeResponse}
\end{question}

Here is a picture of a llama:
\begin{image}
\includegraphics{llama.pdf}
\end{image}

If you like, check out this video\youtube{0aQpLSu2fMs}.




\begin{exploration}
Write a Python script that will compute factorial for you.
\begin{python}
def honest_factorial(x):
  result = 1
  for i in range(1,x+1):
    result *= i
    return result
    
def verifier():
  for i in range(10,20):
    if factorial(i) != honest_factorial(i):
      raise "Your function failed for input " + str(i)
  return True
\end{python}
\end{exploration}

\end{document}

