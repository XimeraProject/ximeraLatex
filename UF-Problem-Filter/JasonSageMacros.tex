
%%%%%%%%%%%%%%%%%%%%%%%%%%%%%%%%%%%%%%%%%%%%%%%%%%%%%%%%%%%%%%%%%
%%%%%%%%%%% Define Sage Commands and Functions %%%%%%%%%%%%%%%%%%
%%%%%%%%%%%%%%%%%%%%%%%%%%%%%%%%%%%%%%%%%%%%%%%%%%%%%%%%%%%%%%%%%
\begin{sagesilent}
#####Define default Sage variables.
#Default function variables
var('x,y,z,X,Y,Z')
#Default function names
var('f,g,h,dx,dy,dz,dh,df')
#Default Wild cards
w0 = SR.wild(0)


#Define a function to get a random computable angle between 0 and 2pi. 
#Syntax: RandAng(AngMin, AngMax) Where Angmin is minimum desired angle, Angmax is maximum desired angle. These are not strict inequalities, be wary.

def RandAng(s,t):
   angle=[0, pi/6,pi/4,pi/3,pi/2, 2*pi/3,3*pi/4,5*pi/6,pi, 7*pi/6,5*pi/4,4*pi/3,3*pi/2, 5*pi/3,7*pi/4,11*pi/6,2*pi ]
   theta=Integer(randint(0,16))
   Ang=angle[theta]
   while Ang<=s or Ang>=t:
      theta=Integer(randint(0,16))
      Ang=angle[theta]
   return simplify(Ang)
   
   
#####Define a sage function to take in a number and return the signed number, so -3 returns "-3" and 3 returns "+3" to interface with Latex correctly. Note this only works with numbers, and if 0 is entered, it returns blank space (To avoid the +0 or -0 result).

var('TempSignVar11')

def ISP(b):
   if b > 0:
      TempSignVar11 ="+" 
   elif b < 0:
      TempSignVar11 ="-"
      b = -b
   else:
      b =""
      TempSignVar11 =""
   b = str(b)
   return TempSignVar11 + b
###################################

# Define a function to substitute a number for a variable in a function without actually evaluating that function.

def NoEval(f, c):
    cstr = str(c)
#    flatex = latex(f)
    fstring = latex(f)
    fstrlist = list(fstring)
    length = len(fstrlist)
    fstrlist2 = range(length)
    for i in range(0, length):
        if fstrlist[i] == "x":
            fstrlist2[i] = "("+cstr+")"
        else:
            fstrlist2[i] = fstrlist[i]
    f2 = join(fstrlist2,"")
    return LatexExpr(f2)

###################################
# Define a function that will convert out hyperbolic expressions and simplify them.

def HyperSimp(f):
    ftemp = f
    ftemp1 = ftemp.substitute(sinh(w0) == (exp(w0) - exp(-w0))/2)
    ftemp2 = ftemp1.substitute(cosh(w0) == (exp(w0) + exp(-w0))/2)
    ftemp6 = ftemp2.substitute(tanh(w0) == (exp(w0) - exp(-w0))/(exp(w0) + exp(-w0)))
#Not working yet    ftemp4 = ftemp3.substitute(sech(w0) == 2/(exp(w0) + exp(-w0)))
#Not working yet    ftemp5 = ftemp4.substitute(csch(w0) == 2/(exp(w0) - exp(-w0)))
#Not working yet    ftemp6 = ftemp5.substitute(coth(w0) == (exp(w0) + exp(-w0))/(exp(w0) - exp(-w0)))
    ftemp7 = ftemp6.substitute(arcsinh(w0) == ln( w0 + sqrt((w0)^2 + 1)))
    ftemp8 = ftemp7.substitute(arccosh(w0) == ln( w0 + sqrt((w0)^2 - 1)))
    ftemp9 = ftemp8.substitute(arctanh(w0) == 1/2 * ln( (1 + w0)/ (1 - w0) ))
    ftemp10 = ftemp9.substitute(arccsch(w0) == ln( (1 + sqrt((w0)^2 + 1))/w0))
    ftemp11 = ftemp10.substitute(arcsech(w0) == ln( (1 + sqrt(1 - (w0)^2))/w0))
    ftemp12 = ftemp11.substitute(arccoth(w0) == 1/2 * ln( (1 + w0)/ (w0 - 1) ))
    ffinal = ftemp12
    return simplify(ffinal)

###################################

# Define a quick version of Integer(randint(-a,b))

def RandInt(a,b):
    return Integer(randint(a,b))



###################################
#Define a function to create an integer from b to c that is not some set of numbers (default is not 0).
## 
## Syntax: To generate a non zero integer between -3 and 17, use NonZeroInt(-3,17).
## To generate a number between -3 and 17, that is not 0, 5, or ROOT1 (Where ROOT1 is a variable calculated earlier in the code) use NonZeroInt(-3,17, [0, 5, ROOT1]).
## 


var('TempVar21')

def NonZeroInt(b,c, D = [0]):
   TempVar21 = Integer(randint(b,c))
   while TempVar21 in D:
      TempVar21 = Integer(randint(b,c))
   return TempVar21
###################################


### Define a Random Nonzero vector in range b to c inclusive
## Syntax: RandomVector(A,B,[LIST], C):
##
## The function will generate a list of numbers from A to B, in randomized order. 
## If you wish to remove any values from the list (such as 0, or the sage number calculated earlier) include it in the LIST area. Note the [], they are necessary. This argument is optional.
## If you wish to duplicate the string (before it is randomized) a certain number of times, put that number in for C. This argument is optional.
## 
## So, for example, if you want to just generate a list of numbers in random order from -5 to 17, use: RandomVector(-5,17).
## If you want to generate a list of random numbers that include -4 to 12, but not 0, -2, or ROOT1 (a variable calculated earlier in the code), and have each number occuring 3 times use: RandomVector(-4,12,[0,-2,ROOT1], 3).

var('TVecFun11')
var('TVecFun12')
var('TVecFun13')
var('TVecFun14')
var('TVecFun15')
import random

def RandomVector(b,c,TVecFun12=[],TVecFun13 = 1):
   TVecFun11 = range(b, c+1)
   TVecFun15 = [p for p in TVecFun11 if p not in TVecFun12]
   TVecFun14 = TVecFun15
   for j in range(1, TVecFun13):
      TVecFun14 = TVecFun14 + TVecFun15
   random.shuffle(TVecFun14)
   return TVecFun14

## Diagonal of the matrix.
def diag(A,dim,B):
   for i in range(dim):
      B[i,i] = A[i,i]
   return B
   
   
## Generate a random QQ matrix
def gen_matrix(dim1, dim2, LBnum,UBnum,LBden,UBden):
   Temp = matrix(QQ, dim1, dim2, 0)
   for i in range(dim1):
      for j in range(dim2):
         num = RandInt(LBnum, UBnum)
         den = RandInt(LBden, UBden)
         Temp[i,j] = num/den
   return Temp
   
   
###################################

\end{sagesilent}

%%%%%%%%%%%%%%%%%%%%%%%%%%%%%%%%%%%%%%%%%%%%%%%%%%%%%%%%%%%%%%%%%
%%%%%%%%%%%%%% End Sage Commands and Functions %%%%%%%%%%%%%%%%%%
%%%%%%%%%%%%%%%%%%%%%%%%%%%%%%%%%%%%%%%%%%%%%%%%%%%%%%%%%%%%%%%%%
