\documentclass{ximera}
%%% You can put user macros here
%% However, you cannot make new environments

\graphicspath{{./}{firstExample/}}

\usepackage{tikz}
\usepackage{tkz-euclide}
\usetkzobj{all}

\tikzstyle geometryDiagrams=[ultra thick,color=blue!50!black]

\title{Numeric Targeted Feedback}

\makeatletter
\define@key{choice}{feedback}[]{\def\choice@feedback{#1}}
%%
\def\HCode{}

\begin{document}
\newcounter{feedbackOutput}
\setcounter{feedbackOutput}{0}

\newcommand{\feedbackOutput}[1]{
    \HCode{<p id="#1">}
    \HCode{</p>}
    }
\begin{abstract}
Generic testing link.
\end{abstract}
\maketitle

\begin{javascript}
//  A validator that also allows for targeted feedback. 
//  Currently only works with numeric answers with some limitations (read comments in the validator). 
//  
//  Moreover, you need to use the corresponding ``\feedbackOutput" command which builds the html for the output.
//  You need the argument of the \feedbackOutput command to match the 'outputName' that you enter in the
//      validator function (see the example; using the output name 'outputOne'. 
//      Note that the apostrophes are necessary as JS needs it to be a string.
//
//  Finally, the input for the validator needs to be written in javascript code, not LaTeX which is a pain.
//      See the note in the validator; in short you need to use Math.e or Math.pi for the math constants.
//      For a list of javascript usable functions go to:
//          https://www.w3schools.com/js/js_math.asp
//      Keep in mind that the math module (all lowercase, as oppose to Math) is not loaded.
//          Among other things, this means there are no complex numbers or complex valued functions.


function feedbackFunc(studentInput,outputName,ansCheck,ansFeedback,defaultfeedback='') {
    var studentAns = studentInput.evaluate();
    
    // Note that you can use e and pi by using Math.E and Math.PI; 
    //      but these are using javascript values, not Ximera equalities.
    // This means testing equality is a lot more fragile; 
    //      in particular function equality will be nearly pointless to test.
    
    // Set the default feedback text, for a student answer that doesn't match any predicted answer.
    var feedbackText = defaultfeedback;
        
    // Now loop through the expected list and see if the student entered an expected answer.
    for (var i = 0; i < ansCheck.length; i++) {
        // If the students answer matches the current expected answer, print the corresponding feedback.
        if (studentAns==ansCheck[i]) {
            feedbackText = 'Feedback: ' + ansFeedback[i];
        }
        // If you have found the correct answer, return correct and we're done.
        if (studentAns==ansCheck[0]) {
            return 1;
            } 
    }
    
    // Print the feedback to the dummy element we built with \feedbackOutput.
    document.getElementById(outputName).innerHTML = feedbackText;
    
    // If we have gotten this far, we haven't found the right answer, so return false.
    return 0;
}
\end{javascript}

\begin{problem}
    Testing new feedback mechanism. The correct answer is $\frac{20}{3}$. But try typing in 4 or $5-e$ first for targeted feedback example. This is to demonstrate that you can use some irrational constants, but keep in mind that they are javascript coded, not latex coded. See tex file for more detail. This validator currently only works with numeric values, and even then only kinda.
    
    % The function ``feedbackFunc'' below is a numeric-only targeted feedback system. In order, the arguments are:
    %feedbackFunc(
    %    inputOne,
    %    'outputOne',
    %    [6+2/3,5-Math.E,4],
    %    ['Correct!','Try going up 1+e+2/3','Try going up 8/3'],
    %    'Feedback: Was Not Expecting That! Try entering four...')
    %
    %   #1  The id of the student input, set in the \answer command you want feedback on (see example).
    %   #2  The output id of the html that will be replaced by feedback; so you can place it where you want.
    %   #3  The vector of expected answers. The first one MUST be the correct answer currently.
    %   #4  The vector of corresponding targeted feedback; should overlay the expected values (see example)
    %   #5  The default feedback in case an answer that isn't in the expected list is entered. Optional.
    %
    %   Also note that you MUST encapsulate the entire function in braces 
    %       to stop tex from breaking the content up. (See example)
    %
    %   Finally; javascript is pretty sensitive to white space and line breaks; try to avoid those if possible.
    %
    \begin{validator}[{feedbackFunc(inputOne,'outputOne',[6+2/3,5-Math.E,Math.PI/6,4],['Correct!','Try going up 1+e+2/3','You entered the arcsin(1/2)','Try going up 8/3'],'Feedback: Was Not Expecting That! Try entering four...')}]
        $\answer[id=inputOne]{x}$.
        % Note that we MUST use \feedbackOutput below to build the html that then is replaced by feedback.
        %   Moreover the argument: outputOne corresponds to 'outputOne' in the function above. This is necessary.
        \feedbackOutput{outputOne}
    \end{validator}
    
\end{problem}




\end{document}
