\documentclass{ximera}
\title{Desmos Embeddings}


\begin{document}
\begin{abstract}
    How to embed Desmos Lessons
\end{abstract}
\maketitle

\subsection*{How's it work?}

    You can embed Desmos lessons into Xronos easily, much like you can with YouTube, using the \verb|\desmos| command. Similarly to the youtube command, Desmos will take in a portion of the url (once the Desmos lesson is marked as embeddable) and configure everything else for you. By way of example, I have a lesson I give student to be able to play around with and understand translations and transformations here: 
    
    Thus if you wanted to share a Desmos lesson at \url{https://www.desmos.com/calculator/ryvd3xeltm}. To embed this, all we need is the string after the last slash, so we would use the command: \\ \verb|\desmos{ryvd3xeltm}{}{500}|\footnote{the second argument is the width - which defaults to the width of the window. The third is the height, which usually defaults to something too small, so I typically use 500 or so.}. Which gives us:

        \desmos{ryvd3xeltm}{}{500}

\subsection*{Validation}
    Playing around with Desmos lessons has no natural ``completion'' metric, so it is currently ungraded - meaning that student interaction doesn't count toward progress on the page in any way.

    
\subsection*{Optional Arguments}
    There are currently no optional arguments for Desmos.

\subsection*{Potential Pitfalls and Problems}
    \subsubsection*{No grade potential}
        Since students can't earn points from it, it can sometimes be difficult to motivate them to interact with a Desmos lesson. One option is to make followup problems that require some kind of manipulation within your Desmos lesson in order to be able to answer them, but finding a motivational method is up to the creativity of the author.
    
\end{document}