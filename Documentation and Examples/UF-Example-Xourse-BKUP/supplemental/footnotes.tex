\documentclass{ximera}
\title{Footnotes}


\begin{document}
\begin{abstract}
    Show that footnotes work without anything special.
\end{abstract}
\maketitle

Footnotes are used frequently in math text, but it isn't clear how they should be dealt with in an online format. Eventually I aim to build out footnotes to work more like \href{https://what-if.xkcd.com/}{the xkcd what-if footnotes}. For now though, footnotes \textit{do} work, and they work by expanding in-line in a different color font.

So, you can use something like:

\begin{verbatim}
    We love to have footnotes in math texts%
    \footnote{Or, at least, they seem to keep getting used...}
    even though they make you go look elsewhere to see what is happening.
\end{verbatim}

Becomes:

We love to have footnotes in math texts%
\footnote{Or, at least, they seem to keep getting used...}
even though they make you go look elsewhere to see what is happening.

Thus there is nothing special about the syntax or usage to let you use footnotes. It is worth a note here though that you may need to update your ximera.cls, ximera.4ht, and possibly a few other files to make this work - but if you are interested in doing so it's a simple matter of copy/pasting the files, just contact the Xronos dev and they will send you updated files.

\end{document}