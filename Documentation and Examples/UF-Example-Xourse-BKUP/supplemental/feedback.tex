\documentclass{ximera}
\title{Feedback} 


\begin{document}
\begin{abstract}
    Demonstration of the 'feedback' environment
\end{abstract}
\maketitle


\subsection{How feedback works}
    There is a feedback environment to give feedback based on if the answer is correct, or if the student just tried something.
    
    \subsubsection*{How to generate/use the feedback environment.}
        
        Feedback can be provided to students using the feedback environment within a problem-style environment. The feedback environment takes some optional arguments (see below) to determine when it is displayed, but by default it is displayed after any attempt at the problem (regardless of whether that attempt was correct or incorrect). 
        
        For example, consider the following:
        
        \begin{problem}
            Guess my number! $\answer{\pi}$
            \begin{feedback}[attempt]
                $\pi$. The answer is $\pi$.
            \end{feedback}
        \end{problem}
        
        The previous problem is generated by the code:
        
        \begin{verbatim}
            \begin{problem}
                Guess a number! $\answer{\pi}$
                \begin{feedback}
                    $\pi$. The answer is $\pi$.
                \end{feedback}
            \end{problem}
        \end{verbatim}
        
        It is also worth noting that you can stack feedback environments in the same question. For example, try the following:
        
        \begin{problem}%%%%
            Ok, ok, that was unfair. Guess an integer this time. $\answer{31415}$
            \begin{feedback}
                The answer is 31415. Obviously.
            \end{feedback}
            \begin{feedback}[correct]
                And this is only displayed because you've entered the correct answer. Good work!
            \end{feedback}
        \end{problem}     
   


    \subsection*{Optional Arguments}
    
        Currently feedback only has two optional arguments, which serve as triggers for when the feedback is displayed:
        
        \begin{description}
            \item[correct] The optional argument ``correct'' will ensure that the feedback is only displayed once the question has been flagged as ``correct'' and locked. This can be used for followup explanation of the problem once it has been correctly answered - in case you want to include information that would be too much information for a student still trying to answer it.
            \item[attempt] The optional argument ``attempt'' will display the feedback after any attempt, regardless of whether that attempt is correct or incorrect. This is the default behavior of feedback if no optional argument is provided, and note that it is the only behavior that gives feedback when a student gives a wrong answer. However, it also doesn't go away once the student gets the correct answer.
        \end{description}
        
        
    \subsection*{Potential Pitfalls and Problems}   
        
        \subsubsection*{Limited triggering}
            The feedback system only has the two current methods of triggering, either a correct answer, or an attempt. There is a (very) limited custom validator written to give targeted feedback based on specific numeric answers, but it uses javascript validation methods, which are ... not awesome. 
            
        \subsubsection*{Content is only hidden, but it exists}
            Technically the content within a feedback environment exists, but is just hidden until triggered. This means, if you want to do something like put another problem within the feedback environment, and you customize the trigger so that a student completes the page without triggering the feedback, then the problem they don't see still counts for ``page credit'' meaning that the student is now stuck without being able to get full credit.
            
            Under the current system, the only way a problem locks is if they get it right, and correctly answered problems \textit{always} display the feedback under either of the default trigger behaviors - meaning that the above concern isn't possible under the existing system. But if you want to play with the system and do custom work, be aware that this is an easy pitfall to land in by accident.
    
    
    
\end{document}