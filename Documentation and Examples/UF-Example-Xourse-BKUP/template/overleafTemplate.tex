\documentclass{ximera}
\title{Overleaf Template}


\begin{document}
\begin{abstract}
    We provide a link to an Overleaf Template that can be copied/saved and used for those without the ability, interest, and/or knowledge to install things locally.
\end{abstract}
\maketitle

This is a link to an overleaf project with some instructions and a backup of the Calc sequence homework files. Note that this isn't necessarily maintained, so to get the most recent files you should visit the actual page and append ``.tex'' to see the actual code in use, but the included code in overleaf gives a decent place to start and look through existing content.

\href{https://www.overleaf.com/read/ptfwfbvzfxjz}{Here is the link to the Overleaf Project Template}. 

Also be aware that overleaf does not support sage, so any randomization should be debugged on a sage (python) server. There are a bunch of public servers available, so Googling something like ``online sage compiler'' should get you access to one to check if your code works. That being said, a local install of LaTeX, sage, and sagetex would be ideal since you can debug everything locally and in-line.

\end{document}