% \subsubsection{Utility macros}

% Improved version of |\newcounter| to check for existance before
% creating a counter to minimize conflicts with packages.  
% Code located in "Utilitymacros.dtx"
% Added by Jason Nowell
%    \begin{macrocode}
%<*classXimera>

\newcommand{\Make@Counter}[1]{
  \@ifundefined{c@#1}% Check to see if counter exists
	       {     % If not, create it and set it to 0.
		\newcounter{#1}
		\setcounter{#1}{0}
		}
		{%If so, reset to 0.
		\setcounter{#1}{0}
		}
}
%    \end{macrocode}

% Added for those that want to use UF problems without using the
% problem filter code. This command is renewed into something
% meaningful in the 'ProblemSelector.sty'.
%    \begin{macrocode}
\providecommand{\latexProblemContent}[1]{#1}
% Iterate count for problem counts.
\Make@Counter{Iteration@probCnt}        
%    \end{macrocode}



% \DescribeEnv{suppress}{
% The suppress command is a good way to suppress output without commenting it.
% This prints/executes code that is being suppressed by printing with "invisible" font.
% This allows commands to execute while suppressing any output.
%
% This way we can avoid many of the places we use environ package and this should also
% avoid most of the verbatim conflicts. This is only useful for the pdf since the html
% conversion eats all the formating.
%
% Code located in "Utilitymacros.dtx"
% Jason Nowell -- Code adapted from syntonly.sty
%    \begin{macrocode}
\font\dummyft@=dummy \relax
\def\suppress{%
  \begingroup\par
  \parskip\z@
  \offinterlineskip
  \baselineskip=\z@skip
  \lineskip=\z@skip
  \lineskiplimit=\maxdimen
  \dummyft@
  \count@\sixt@@n
  \loop\ifnum\count@ >\z@
    \advance\count@\m@ne
    \textfont\count@\dummyft@
    \scriptfont\count@\dummyft@
    \scriptscriptfont\count@\dummyft@
  \repeat
  \let\selectfont\relax
  \let\mathversion\@gobble
  \let\getanddefine@fonts\@gobbletwo
  \tracinglostchars\z@
  \frenchspacing
  \hbadness\@M}
\def\endsuppress{\par\endgroup}
%</classXimera>
%    \end{macrocode}
