% \subsubsection{Multiple choice and the like}
% \DescribeEnv{multipleChoice}{Multiple choice}
%    \begin{macrocode}
%<*classXimera>
% Jim: Originally this was \renewcommand{\theenumi}{$(\mathrm{\alph{enumi}})$}
% but that breaks tex4ht because mathmode can only be processed by mathjax.
% so now I made this just italicized.
%    \end{macrocode}
% \subsubsection{Options}
%    \begin{macrocode}
\define@key{choice}{value}[]{\def\choice@value{#1}}
%    \end{macrocode}
% This flags the answer as the correct answer
%    \begin{macrocode}
\define@boolkey{choice}{correct}[true]{\def\choice@correct{#1}}
%    \end{macrocode}
% Use an ID to refer to the choice.
%    \begin{macrocode}
\define@key{multipleChoice}{id}{\def\mc@id{#1}}
%    \end{macrocode}
% |\otherchoice| outputs the item if correct and nothing if incorrect.
%    \begin{macrocode}
\define@key{otherchoice}{value}[]{\def\otherchoice@value{#1}}
\define@boolkey{otherchoice}{correct}[true]{\def\otherchoice@correct{#1}}
%    \end{macrocode}
% Default key choices for multiple choice options.
% Default for choice pairs.
% Default: answers without the option "correct=true" is "incorrect".
%    \begin{macrocode}
\setkeys{choice}{correct=false,value=}
%    \end{macrocode}
% Defaults for multipleChoice pairs.
% Default to no id? -- Jason
%    \begin{macrocode}
\setkeys{multipleChoice}{id=}
%    \end{macrocode}
% Defaults for otherchoice pairs.
% Default "otherchoice" to behave like "choice" for error checking.
%    \begin{macrocode}
\setkeys{otherchoice}{correct=false,value=}
%</classXimera>
%    \end{macrocode}
%
% \subsubsection{Choices}
% \DescribeMacro{\choice}{Like |\item| but for choice environments.}
% choice command denotes a possible answer choice for the multiple choice question.
%    \begin{macrocode}
%<*classXimera>
\newcommand{\choice}[2][]{%
\setkeys{choice}{#1}%
\item{#2}
\ifthenelse{\boolean{\choice@correct}}
    {% Begin then result
    \ifhandout% if it's a handout do nothing.
    \else% otherwise place a checkmark when you select the "correct choice"... maybe? -- Jason
        \,\checkmark\,\setkeys{choice}{correct=false}
    \fi
    }% End then result
    {}% Begin/End else result.
}

%Define an expandable version of choice Not really meant to be used outside this package (used in wordChoice)
% Is there a reason we can't just always use this as default? -- Jason
\newcommand{\choiceEXP}[2][]{%
	\expandafter\setkeys\expandafter{choice}{#1}%
	\item{#2}
	\ifthenelse{\boolean{\choice@correct}}
		{% Begin then result
		\ifhandout 
		\else 
			\,\checkmark\,\setkeys{choice}{correct=false}
		\fi
		}% End then result
		{}% Begin/End else result.
	} %% note all the {} are needed in case the choice has [] in it.

% \otherchoice is the \choice used in wordChoice command.
\newcommand{\otherchoice}[2][]{%
\ignorespaces%
\setkeys{otherchoice}{#1}%
\ifthenelse{\boolean{\otherchoice@correct}}%
{% Start then result
#2\ignorespaces\setkeys{otherchoice}{correct=false}\ignorespaces%
}% End then result
{}% Start/End else result
\ignorespaces%
}%

\newcommand{\inlinechoice}[2][]{%
\setkeys{choice}{#1}%
\iffirstinlinechoice
(\hspace{-.25em}
\firstinlinechoicefalse
\else
/
\fi
#2
\ifthenelse{\boolean{\choice@correct}}%
{% Start then result
\ifhandout\else\checkmark\ignorespaces\setkeys{choice}{correct=false}\ignorespaces\fi%
}% End then result
{}% Start/End else result
\hspace{-.25em}\ignorespaces%
}

%</classXimera>
%    \end{macrocode}
% On the HTML side, |\choice| emits |<span>|s.
%    \begin{macrocode}
%<*htXimera>
\newcounter{choiceId}
\renewcommand{\choice}[2][]{%
\setkeys{choice}{correct=false}%
\setkeys{choice}{#1}%
\stepcounter{choiceId}\IgnorePar%
\HCode{<span class="choice }%
\ifthenelse{\boolean{\choice@correct}}{\HCode{correct}}{}
\HCode{" }
\ifthenelse{\equal{\choice@value}{}}{}{\HCode{data-value="\choice@value" }}
\HCode{id="choice\arabic{choiceId}">}%
#2\HCode{</span>}}
\let\inlinechoice\choice
%</htXimera>
%    \end{macrocode}

% \subsubsection{Environment(s)}
% The environment |multipleChoice@| is for internal use only. 
% \DescribeEnv{multipleChoice}{Wrap |\choice|s in a |multipleChoice| environment to make a multiple choice question.}
%    \begin{macrocode}
%<*classXimera>
\newenvironment{multipleChoice}[1][]
	{% Environment Start Code
	\setkeys{multipleChoice}{#1}%
	\recordvariable{\mc@id}%
	\begin{trivlist}
		\item[\hskip \labelsep\small\bfseries Multiple Choice:]\hfil
		\begin{enumerate}
	}% Note this means that \item has to be the first line after \begin{multipleChoice}.
	{% Environment End Code
		\end{enumerate}
	\end{trivlist}
	}

%multipleChoice@ is for internal use only! (used in wordChoice)
%this is simply a wrapper for the sole showing (other)choice.
\newenvironment{multipleChoice@}[1][]{}{)}
%</classXimera>
%    \end{macrocode}



% On the web, you might also expect these to be "problem environments"
% but they aren't -- they're respondables.  You might expect a
% |\setcounter{choiceId}{0}| here --- that would be wrong, because then
% the generated IDs would no longer be unique.
%    \begin{macrocode}
%<*htXimera>
\renewenvironment{multipleChoice}[1][]
{\setkeys{multipleChoice}{#1}%
\stepcounter{identification}\ifvmode \IgnorePar\fi \EndP\HCode{<div class="multiple-choice" }%
\ifthenelse{\equal{\mc@id}{}}{}{\HCode{data-id="\mc@id" }}%
\HCode{id="problem\arabic{identification}">}%
}{\HCode{</div>}\IgnoreIndent}
\ConfigureEnv{multipleChoice}{}{}{}{}
%</htXimera>
%    \end{macrocode}

% \subsection{Word choice}
% \DescribeMacro{\wordChoice}{An in-line version of multipleChoice:
% uses enumitem package note, it is coded as a single line to avoid
% unwanted spaces in ``given'' mode.}
%    \begin{macrocode}
%<*classXimera>
\newcommand{\wordChoice}[1]{%
\let\choicetemp\choice% Assign a "choicetemp" command to duplicate choice.
\ifwordchoicegiven% If wordchoice option is on, we need to juggle around some definitions.
\let\choice\otherchoice%
\begin{multipleChoice@}%
#1
\end{multipleChoice@}%
\else% If it isn't the regular "choice" command should work.
\let\choice\inlinechoice%
\begin{multipleChoice@}%
#1%
\end{multipleChoice@}%
\fi%
\let\choice\choicetemp% Now that choicetmp has been manipulated to what we want, replace choice with it.
}%


%</classXimera>
%    \end{macrocode}
% This is actually just word choice
%    \begin{macrocode}
%<*htXimera>
\renewenvironment{multipleChoice@}{\refstepcounter{problem}}{}%
\ConfigureEnv{multipleChoice@}{\stepcounter{identification}\IgnorePar\HCode{<span class="word-choice" id="word-choice\arabic{identification}">}}{\HCode{</span>}\IgnoreIndent}{}{}
%</htXimera>
%    \end{macrocode}

% \subsection{Select all}
% \DescribeEnv{selectAll}{A multiple-multiple choice question}
%    \begin{macrocode}
%<*classXimera>
\newenvironment{selectAll}[1][]
{\begin{trivlist}\item[\hskip \labelsep\small\bfseries Select All Correct Answers:]\hfil\begin{enumerate}}
    {\end{enumerate}\end{trivlist}}
%</classXimera>
%    \end{macrocode}

% In the future we need this to (optionally) be displayed in the
% problem, while the actual code lives in the solution. Here is how
% this could be implemented: Like the title/maketitle commands, the
% multiple-choice could be stored in |\themultiplechoice|, flip a
% boolean, and execute |\makemultiplechoice| at the |\end| of the
% problem.  We should also make a command called |\showchoices| that
% will show choices in the handout.

% On the web, |selectAll| is handled just like |multipleChoice|.
%    \begin{macrocode}
%<*htXimera>
\renewenvironment{selectAll}{\refstepcounter{problem}}{}%
\ConfigureEnv{selectAll}{\stepcounter{identification}\ifvmode \IgnorePar\fi \EndP\HCode{<div class="select-all" id="problem\arabic{identification}">}}{\HCode{</div>}\IgnoreIndent}{}{}
%</htXimera>
%    \end{macrocode}


