%    \begin{macrocode}
%<*classXimera>
%%%%%%%%%%%%%%%%%%%%%%%%%%%%%%%%%%
%%%%%%%%%%%%%%%%%%%%% Validator
%%%%%%%%%%%%%%%%%%%%%%%%%%%%%%%%%%

% New Validator rewrite code added by Jason Nowell. Original code
%provided by Jim Fowler Validator is an environment designed to run a
%custom check on answers (usually) using javascript code.


% Define a placeholder command for validator and feedback.

\newcommand{\PH@Command}{}

%% Validator should take an argument and detokenize it and display it
%% at the start of the environment. The original Validator environment
%% had everything framed in an mbox; presumably to make the text look
%% a bit nicer, although this seems redundant with texttt. It
%% shouldn't cause any harm so I have left it in for now.

\newenvironment{validator}[1][]{
	\def\PH@Command{#1}% Use PH@Command to hold the content and be a target for "\expandafter" to expand once.
	\mbox{\texttt{\detokenize\expandafter{\PH@Command}}}% Now expand PH@Command once and then detokenize. 
	}{}

%%%% Feedback

\ifhandout%	First, if it's a handout, we want feedback to eat everything and then disappear entirely. So we use the environ package to eat the content by not using \BODY command anywhere.
\NewEnviron{feedback}[1][]{}

% No! We are going to replace with the suppress command method. Delete the above old code by 6/28/2018 if no errors
%\newenvironment{feedback}{\expandafter\phantom\expandafter\begingroup\expandafter\noexpand\begingroup}{\endgroup\endgroup}

\else% If this isn't a handout, then we want to display the Feedback
     %by using a label, positioned and formated as a \item in a
     %trivlist. It is important that we also detokenize the content of
     %the optional argument, as it is likely to contain javascript or
     %other code that latex won't be able to make sense of.

\newenvironment{feedback}[1][attempt]{

	\def\PH@Command{#1}% Use PH@Command to hold the content and be a target for "\expandafter" to expand once.
	
	\begin{trivlist}% Begin the trivlist to use formating of the "Feedback" label.
	\item[\hskip \labelsep\small\slshape\bfseries Feedback% Format the "Feedback" label. Don't forget the space.
	(\texttt{\detokenize\expandafter{\PH@Command}}):% Format (and detokenize) the condition for feedback to trigger
	\hspace{2ex}]\small\slshape% Insert some space before the actual feedback given.
	}{
	\end{trivlist}	
	}

\fi

%</classXimera>
%    \end{macrocode}


%    \begin{macrocode}
%<*htXimera>
%%%%%%%%%%%%%%%%%%%%%%%%%%%%%%%%%%%%%%%%%%%%%%%%%%%%%%%%%%%%%%%%
% Feedback environments take an optional parameter (which describes
% when the feedback is to be provided)

\def\feedback{\@ifnextchar[{\@feedbackcode}{\@feedbackattempt}}
\def\@feedbackattempt{\@feedbackcode[attempt]}
\def\@feedbackcode[#1]{\stepcounter{identification}%
\ifvmode \IgnorePar\fi \EndP%
\ifthenelse{\equal{#1}{attempt}}{\HCode{<div class="feedback" data-feedback="attempt" id="feedback\arabic{identification}">}}%
{\ifthenelse{\equal{#1}{correct}}{\HCode{<div class="feedback" data-feedback="correct" id="feedback\arabic{identification}">}}%
{\HCode{<div class="feedback" data-feedback="script" id="feedback\arabic{identification}"><script type="text/javascript">\Hnewline//<![CDATA[\Hnewline function feedback\arabic{identification}() }\{\HCode{ return \detokenize{#1}; }\}\HCode{\Hnewline //]]>\Hnewline</script>}}}}
\def\endfeedback{\HCode{</div>}\IgnoreIndent}
%</htXimera>
%    \end{macrocode}
