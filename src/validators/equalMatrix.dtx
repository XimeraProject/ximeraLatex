% \subsubsection{equalMatrix}
%   This validator implements a way for students to enter in a matrix answer, 
%       and have that compared against an instructor's matrix answer entry-by-entry
%       using the ximera equality algorithm (so this supports algebraic entries).
%   It expects a matrix to be defined using the following formatting:
%       [ [r1c1,r1c2,r1c3],[r2c1,r2c2,r2c3],[r3c1,r3c2,r3c3] ]
%       Where "rXcY" is the "row X, column Y" entry of the matrix.
%   Thus both the instructor and the student should input this format for the matrix.
%       So the instructor would write something like:
%           $\answer{[ [1,2,3],[4,5,6],[7,8,9] ]}$
%       And the student would click into the answer box and enter in:
%           [ [1,2,3],[4,5,6],[7,8,9] ]
%       This has the (somewhat advantage) of being parsed into something visible by mathjax,
%           rather than the 'invalid symbol' or 'unexpected token' or other error, which is nice.
%
%    \begin{macrocode}
%<*classXimera>
%% Apend code via \@addToValidators

\appendtoverbtoks?
// We want to define a validator that allows us to check a matrix.
function equalMatrix(f,g) {
    /* We expect that f is a matrix provided by the student in the format:
            [ [r1c1,r1c2,r1c3],[r2c1,r2c2,r2c3],[r3c1,r3c2,r3c3] ]
       We expect that g is a matrix provided by the instructor in the format:
            [ [r1c1,r1c2,r1c3],[r2c1,r2c2,r2c3],[r3c1,r3c2,r3c3] ]
    */
    
    
    debugText('Here is the student ans.');
    
    debugArray(f);
    
    debugText('Here is the instructor ans.');
    
    debugArray(g);
    
    //  Now we want to make a dummy form of the given variables so we can manipulate them without the original getting recked.
   
    var fTree = JSON.parse(JSON.stringify(f.tree));
    var gTree = JSON.parse(JSON.stringify(g.tree));
    
    
    // Start out by making sure that the format was entered correctly
    //  in particular, that each sub-list is the same length.
    //instructorAnsbkup = JSON.parse(JSON.stringify(g.tree));
    
    // Santize the instructor's entry.
    
    for (var i = 1; i < g.tree.length; ++i) {
        if (g.tree[1].length==g.tree[i].length) {
        } else {
            console.log('Looks like the format for the instructors answer is off, [at least] one of the rows has the wrong number of columns!');
            return false
        }
    }

    // Santize the student entry.
    for (var i = 1; i < f.tree.length; ++i) {
        if (f.tree[1].length==f.tree[i].length) {
        } else {
            console.log('Looks like the format for the students answer is off, [at least] one of the rows has the wrong number of columns!');
            return false
        }
    }
    
    // Now that we have santized the column/rows so that we know we have an actual rectangle, we'll check that the dimensions match.
    
    if (f.tree.length!==g.tree.length) {
        // If the dimensions aren't the same, then the answer definitely doesn't match the instructors.
        debugText('Looks like the matrix submitted by the student does not have the same dimension as the matrix we expect.');
        return false;
    }
    
    for (var i = 1; i < f.tree.length; ++i) {
        if (f.tree[i].length==g.tree[i].length) {
        } else {
            debugText('Looks like the matrix submitted by the student does not have the same dimension as the matrix we expect.');
            return false
        }
    }
    
    // Finally, we check the actual entries. 
    // We assume that the matrix is a two dimensional array, i.e. no matrix entries inside our matrix. Tensors can bite me.
    
    for (var i = 1; i < f.tree.length; ++i) {
        for (var j = 1; j < f.tree[i].length; ++j) {
            debugText('Checking equality of entries in: row ' + i +', column ' + j);
            
            if ( compSubTree(f,fTree[i][j],gTree[i][j]) ) {
                debugText('They appear equal in student and instructors matricies.');
            } else {
                console.log('Looks like [at least one] of the entries are wrong!');
                debugText('The students answers [raw ans then copied obj ans]')
                debugArray(f.tree[i][j]);
                debugArray(fTree);
                debugText('The students answers [raw ans then copied obj ans]')
                debugArray(g.tree[i][j]);
                debugArray(gTree);
                return false
            }
        }
    }
    
return true
}
?
%
%
%% Now we add a prompt in the answer command to trigger this validator. In particular, we add "validator=equalMatrix" to trigger this validator.
%</classXimera>
%    \end{macrocode}

