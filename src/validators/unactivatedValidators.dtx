% \subsection{unactivatedValidators}
%%%   This section contains a bunch of example validators that aren't actually implemented/used, but
%%%       contain code that may be of use/interest to future developers or may be useful to authors.
%%%   To activate any of the below validators, simply uncomment
%%%       from the \verb|\appendtoverbtoks| to the questionmark at the end and then rebuild the cls file.
%%%   Note that the questionmark at the beginning and end IS NECESSARY, 
%%%       this isn't a comment, it is a token for verbatim content.
%%%       Even though it looks weird, make sure you uncomment it, and do NOT remove or change it.
%%%
% \subsubsection{sameParity}
%%%
%%%   sameParity validator compares the number supplied by instructor and student to see if they
%%%       are both even or both odd. It validates as correct if the two answers are the same (even/odd)
%%%
%%%   To use this validator (once activated), use \verb|\answer[validator=sameParity]{instructorAns}| where
%%%       a number is supplied as the instructorAns.
%%%
%    \begin{macrocode}
%<*classXimera>
%%% Apend code via \verb|\appendtoverbtoks|

%\appendtoverbtoks?
%//<![CDATA[
%%%// NOTE: The below are intended to be used inside an \verb|\answer| optional argument with the validator key, NOT in a validator environment.
%
%// Check to see if two inputs are the same parity in terms of even/odd.
%  function sameParity(a,b) {
%    return (a-b)%2 == 0;
%  };
%//]]>
%?
%
%</classXimera>
%    \end{macrocode}
%%%%
%%%%
% \subsubsection{isPositive}
%%%
%%%   isPositive validates if a number supplied by the student is positive. No instructor answer is necessary.
%%%
%%%   To use this validator (once activated), use \verb|\answer[validator=isPositive]{instructorAns}| where
%%%       the 'instructorAns' code isn't used in any way, so it can actually be left blank.
%%%
%    \begin{macrocode}
%<*classXimera>
%%% Apend code via \verb|\appendtoverbtoks|

%\appendtoverbtoks?
%//<![CDATA[
%var x
%// Check to see if input is positive.
%  function isPositive(number) {
%    return number > 0;
%  };
%//]]>
%?
%
%</classXimera>
%    \end{macrocode}
%%%%
%%%%
% \subsubsection{caseInsensitive}
%
%%%   caseInsensitive validator compares the answer supplied by instructor and student to see if they
%%%       are the same as strings, ignoring case sensitivity.
%%%   Thus if the instructor supplies 'Correct' as the desired string, then
%%%       the following student answers would be validated as correct answers:
%%%       1) CORRECT , 2) correct , 3) CoRReCT , 4) cOrrEct
%%%
%%%   To use this validator (once activated), use \verb|\answer[validator=caseInsensitive]{instructorAns}| where
%%%       the instructorAns (and student's input) are both parsed as strings and compared.
%%%
%    \begin{macrocode}
%<*classXimera>
%%% Apend code via \verb|\appendtoverbtoks|

%\appendtoverbtoks?
%//<![CDATA[
%// NOTE: The below are intended to be used inside an \answer optional argument with the validator key, NOT in a validator environment.
%
%// Check to see if two strings match in a case-insensitive way.
%  caseInsensitive = function(a,b) {
%    return a.toLowerCase() == b.toLowerCase();
%  };
%//]]>
%?
%
%</classXimera>
%    \end{macrocode}

