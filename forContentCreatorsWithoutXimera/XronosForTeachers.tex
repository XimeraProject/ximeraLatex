\documentclass{article}
\usepackage{CustomMacros}

\title{Xronos for Teachers}
\author{Jason Nowell}

\begin{document}
\maketitle
\tableofcontents


\newpage
\section{For the Instructor}
\subsection{Requesting Xronos for your course}
To request Xronos to be used in your course, please contact one of the Xronos Team at the email address: Xronos@math.ufl.edu, or contact Kevin Knudson at KKnudson@ufl.edu, who is the faculty contact for the Xronos Project. Keep in mind that setting up Xronos takes time, especially if you wish to have your own problems included in the assignment. To this end it is \textit{highly suggested} you do the following beforehand:
\begin{itemize}
\item Have a look at the \href{https://xronos.clas.ufl.edu/refcoursenew/Reference}{reference course} beforehand to see which problems in our database you wish to include in your course.
\item Write (in a standard tex file) any additional problems you would like in your homework using \href{TBD}{our template}
\item Determine how many grades you wish Xronos to have in the gradebook, and how many due dates you will need (This will determine how many separate activities the Xronos Team has to build).
\end{itemize}

Once you have considered the above, you should include any information and/or files you've written in your email to the Xronos Team in order for us to get working on them as soon as possible. You should give the assignments to the Xronos Team \textbf{a minimum of 3 weeks before the semester starts} to give them time to get a published version up for you to check.

\subsection{Setting up Xronos in Canvas}
For the most part Xronos works on a by-assignment basis within Canvas like most assignments. Xronos must be added as an external tool by the Xronos team, but once it has been added you can go to the Assignments page and add a new assignment. From there you want to select the "Submission Type" and choose the "External Tool" option. This will bring up another box that will tell you to "Enter or Find External Tool". You want to click on the "Find" button and then navigate to the correct Xronos assignment in the resulting list.

\subsubsection{Individual Xronos Assignments}
Xronos has the ability to add "Tiles", each with a title and each containing an assignment or activity of some form. Currently although all the tiles are individual assignments from the student's standpoint, the group comprises one grade within Canvas (See \textit{Individual Xronos Assignment Grades in Canvas} below). This means that there may be different "assignments", each with their own group of tiles that comprise the actual individual assignments for the students. Keep this in mind when you are considering how to design your Canvas assignments.


\subsubsection{Individual Xronos Assignment Grades in Canvas}
Xronos currently needs to use a different external tool for each "graded assignment" that Canvas uses. This has to do with how the LTI interface works at a core level in Canvas itself, so it is not (currently) something that can be changed in Xronos. Once Canvas updates some of the LTI protocol, Xronos may be able to support more than one grade per Xronos "assignment".

What this means for you, is that if you want different graded assignments for students, they need to be separated into different "tile clusters". Thus your Xronos homework may have several links, each of which going to a different cluster of "Tiles" and each such cluster is it's own grade. 

Since things like due dates and point values of any given canvas assignment is controlled by the assignment itself, this means that if you want multiple due dates or multiple grades for a given "assignment", then it must be broken into different "assignments" within canvas as well. 


\subsection{Writing Xronos Questions}
The Xronos Team has created a template that should allow you to write and compile a version of Xronos problems without needing to download any extra packages or software. This is a limited utility version of the much more powerful Xronos system, so if you would like more advanced features (such as randomization, dynamic feedback, or more complicated validation code) you should contact the Xronos Team directly to discuss such features, but starting from a basic pre-written problem will greatly simplify the process.


\subsection{Xronos During the Semester}

\subsection{Updates}
It is possible to update Xronos pages during the semester. However, it is important to keep in mind that doing so will \textbf{erase all student work on that page} when they update. So, unless there is a truly catastrophic error in the page, it is better to add a new page, tile, or assignment with the additional content. Also keep in mind that \textit{updating is currently up to the student}; meaning that they have to click the update button (which will warn them that doing so will erase all work on the current page). Even if they update and erase all their work, their grade cannot go down however. See "Grading" below for more information on this.

\subsection{Grading}
Grades from Xronos is based upon completion, with point values scaling based upon the assignment's point value set in the Canvas assignment. \textbf{Xronos will only ever increase the completion of a student}, meaning that it is a monotonicly increasing function of time. So if the student does something that eliminates their work (such as the "try another" button, "erase" button, or "update" button) they will not lose any of their current grade. However, since it is based on completion \textit{of the assignment} and \textit{not on completion of individual questions} this also means they must surpass the percentage completed that they had before they reset their work before they start gaining points for the assignment \textit{regardless of which questions they answer}.


\section{For the Students}
\subsection{Grading in canvas}
Xronos has an identity system on it's own. A student can see if they are logged in by checking in the top right of the page to see if it says "Log in" or it has their name. This can be a bit misleading however, because one can be logged into Xronos without being logged into Canvas. This is a problem, because the student's computer needs to form an LTI link between canvas and Xronos in order for their gradebook to be updated. What this means is that the student \textit{\textbf{must log into Xronos from Canvas}} in order to have their work graded. If they bookmark the Xronos site and go to it directly, canvas will not form that LTI link that is made by clicking through canvas. As a result they can do a bunch of work, and then not have the actual work get uploaded to the canvas gradebook, ie they will not receive credit for their work.
Since Xronos has it's own identity system however, once they have logged into Xronos via canvas once, it will automatically sign them in. But this means that if they did go to Xronos directly (and thus incorrectly) they will see that they are logged in, in the top right corner, and it will have all their saved work from when they last logged in, but \textbf{none of their efforts will be saved to canvas}. So it will \textit{seem} like they are getting credit for their work and that everything will be working correctly, but they won't. So make sure to stress this to the students.

\subsection{The ``Try Another" Feature}
Some Xronos Assignments offer a ``Try Another" button at the top of the page. Clicking this button \textit{will erase all your current work}, but it \textit{will not change your current grade}. However, in order to gain any more points on the assignment, you will need to complete the same \textit{percentage} of problems you had completed earlier; the specific problems don't matter. This means, if you hit the ``Try another" button when you are 80\% done with an assignment, you will have to answer another 80\% of the questions before you \textit{start} to get any more points for the assignment. For this reason \textbf{we suggest you only use the ``Try Another" feature once you have attained 100\% on the current page}.

\subsection{``Log" versus ``ln"}
Xronos adopts the industry standard system of using ``$\log$" instead of ``$\ln$" to denote the natural log. Thus, \textbf{when you see ``$\log$", it should always be interpreted as the natural log} and any other base will be denoted using the standard subscript, eg log base ten would be denoted ``$\log_{10}$".




\section{Known bugs or issues}

\subsection{Unexpected function types in answer boxes}
Occasionally, due to how Sage generates randomized problems, functions can appear in the "answer" field that are unexpected; such as hyperbolic trigonometry. It is important to note that, because of how Xronos tests student answers, the actual functions in the "answer" are not necessary to use, as \textbf{Xronos will correctly detect any mathematically equivalent answer} and mark it correct unless told otherwise. For this reason, it is rarely the case that these anomalous functions appear in the answer box actually impact the student in any way. Nonetheless, if there are issues with them, then steps can be taken to correct these problems if you contact the Xronos Team.





\end{document}