% \iffalse meta-comment
% vim: textwidth=75
%<*internal>
\iffalse
%</internal>
%<*readme>
|
------------:| ------------------------------------------------------------
      ximera:| Simultaneously write print and online interactive materials
      Author:| Jim Fowler and Oscar Levin and Jason Nowell and Hans Parshall and Bart Snapp
      E-mail:| bart.snapp@gmail.com
     License:| Released under the LaTeX Project Public License v1.3c or later
         See:| http://www.latex-project.org/lppl.txt

Short description:
Some text about the class: probably the same as the abstract.
%</readme>
%<*internal>
\fi
\def\nameofplainTeX{plain}
\ifx\fmtname\nameofplainTeX\else
  \expandafter\begingroup
\fi
%</internal>
%<*install>
\input docstrip.tex
\keepsilent
\askforoverwritefalse
\preamble
------------:| ------------------------------------------------------------
      ximera:| Simultaneously writing print and online interactive materials
      Author:| Jim Fowler and Oscar Levin and Jason Nowell and Hans Parshall and Bart Snapp
      E-mail:| bart.snapp@gmail.com
     License:| Released under the LaTeX Project Public License v1.3c or later
         See:| http://www.latex-project.org/lppl.txt

\endpreamble
\postamble

Copyright (C) 2018-2020 by Bart Snapp <bart.snapp@gmail.com> and Jim Fowler <kisonecat@gmail.com>

This work may be distributed and/or modified under the conditions of
the LaTeX Project Public License (LPPL), either version 1.3c of this
license or (at your option) any later version.  The latest version of
this license is in the file:

http://www.latex-project.org/lppl.txt

This work is "maintained" (as per LPPL maintenance status) by
Bart Snapp. The source code can be found at:

https://github.com/XimeraProject/ximeraLatex

This work consists of the file ximera.dtx, the associated files
under src/, and a Makefile.

Running "make" generates the derived files README, ximera.pdf, ximera.cls, xourse.cls.

Running "make inst" installs the files in the user's TeX tree.

Running "make install" installs the files in the local TeX tree.

\endpostamble

\usedir{tex/latex/ximera}
\input docstrip
\askforoverwritefalse
\generate{
  \file{ximera.cls}{
    \from{ximera.dtx}{classXimera}
    \from{src/banner.dtx}{classXimera}    
    \from{src/options.dtx}{classXimera}
    \from{src/packages.dtx}{classXimera}
    \from{src/hyperref.dtx}{classXimera}
    \from{src/paragraphs.dtx}{classXimera}
    \from{src/suppress.dtx}{classXimera}
    \from{src/makeCounter.dtx}{classXimera}
    \from{src/pagesetup.dtx}{classXimera}
    \from{src/metadata.dtx}{classXimera}
    \from{src/outcomes.dtx}{classXimera}
    \from{src/macros.dtx}{classXimera}
    \from{src/theorems.dtx}{classXimera}
    \from{src/enumerate.dtx}{classXimera}
    \from{src/proof.dtx}{classXimera}    
    \from{src/only.dtx}{classXimera}
    \from{src/problem.dtx}{classXimera}
    \from{src/hints.dtx}{classXimera}
    \from{src/abstract.dtx}{classXimera}
    \from{src/title.dtx}{classXimera}
    \from{src/image.dtx}{classXimera}
    \from{src/interactives/javascript.dtx}{classXimera}
    \from{src/interactives/include.dtx}{classXimera}
    \from{src/interactives/geogebra.dtx}{classXimera}
    \from{src/interactives/desmos.dtx}{classXimera}
    \from{src/interactives/google.dtx}{classXimera}            
    \from{src/interactives/graph.dtx}{classXimera}    
    \from{src/link.dtx}{classXimera}    
    \from{src/interactives/video.dtx}{classXimera}    
    \from{src/answer.dtx}{classXimera}
    \from{src/choice.dtx}{classXimera}
    \from{src/solution.dtx}{classXimera}
    \from{src/feedback.dtx}{classXimera}
    \from{src/freeresponse.dtx}{classXimera}
    \from{src/verbatim.dtx}{classXimera}
    \from{src/footnote.dtx}{classXimera}    
    \from{src/dialogue.dtx}{classXimera}
    \from{src/instructornotes.dtx}{classXimera}
    \from{src/xkcd.dtx}{classXimera}
    \from{src/foldable.dtx}{classXimera}
    \from{src/leash.dtx}{classXimera}
    \from{src/interactives/sagemath.dtx}{classXimera}        
    \from{src/jax.dtx}{classXimera}
    \from{src/ungraded.dtx}{classXimera}
    \from{src/sectioning.dtx}{classXimera}    
  }
}
\generate{
  \file{xourse.cls}{
    \from{ximera.dtx}{classXourse}
    \from{src/optionsxourse.dtx}{classXourse}
    \from{src/activity.dtx}{classXourse}
    \from{src/sectioning.dtx}{classXourse}
    \from{src/logo.dtx}{classXourse}
    \from{src/graded.dtx}{classXourse}        
  }
}
\generate{
  \file{pgfsys-ximera.def}{
    \from{ximera.dtx}{pgfsys}
    \from{src/pgfsys.dtx}{pgfsys}
  }
}
%</install> 
%<install>\endbatchfile
%<*internal>
\usedir{source/latex/ximera}
\generate{
  \file{ximera.ins}{\from{\jobname.dtx}{install}}
}
\nopreamble\nopostamble
\usedir{doc/latex/ximera}
\generate{
  \file{README.txt}{\from{\jobname.dtx}{readme}}
}
\ifx\fmtname\nameofplainTeX
  \expandafter\endbatchfile
\else
  \expandafter\endgroup
\fi
%</internal>
% \fi
%
% \iffalse
%<*driver>
\ProvidesFile{ximera.dtx}
%</driver>
%<classXimera|classXourse>\NeedsTeXFormat{LaTeX2e}[1999/12/01]
%<classXimera>\ProvidesClass{ximera}
%<classXourse>\ProvidesClass{xourse}
%<classXimera|classXourse> [2020/01/15 v2.00
%<classXimera> Simultaneously write print and online interactive materials]
%<classXourse> Combining Ximera activities into Xourses]
%<*driver>
\documentclass{ltxdoc}
\usepackage[a4paper,margin=25mm,left=50mm,nohead]{geometry}
\usepackage[numbered]{hypdoc}
\usepackage{hyperref}

\EnableCrossrefs
\CodelineIndex
\RecordChanges
\begin{document}
\DocInput{\jobname.dtx}
\section{ximera.cls}
\DocInput{src/options.dtx}
\DocInput{src/packages.dtx}
\DocInput{src/pagesetup.dtx}
\subsection{Structure}
\DocInput{src/paragraphs.dtx}
\DocInput{src/macros.dtx}
\DocInput{src/theorems.dtx}
\DocInput{src/enumerate.dtx}
\DocInput{src/proof.dtx}
\DocInput{src/problem.dtx}
\DocInput{src/hints.dtx}
\DocInput{src/solution.dtx}
\DocInput{src/footnote.dtx}
\DocInput{src/verbatim.dtx}
\DocInput{src/dialogue.dtx}
\DocInput{src/instructornotes.dtx}
\DocInput{src/only.dtx}
\DocInput{src/foldable.dtx}
\DocInput{src/leash.dtx}
\subsection{Document metadata}
\DocInput{src/metadata.dtx}
\DocInput{src/abstract.dtx}
\DocInput{src/title.dtx}
\DocInput{src/outcomes.dtx}
\DocInput{src/labels.dtx}
\subsection{Images}
\DocInput{src/image.dtx}
\DocInput{src/xkcd.dtx}
\subsection{Links}
\DocInput{src/hyperref.dtx}
\subsection{Interactives}
\DocInput{src/interactives/include.dtx}
\DocInput{src/interactives/google.dtx}
\DocInput{src/interactives/geogebra.dtx}
\DocInput{src/interactives/desmos.dtx}
\DocInput{src/interactives/graph.dtx}
\DocInput{src/interactives/video.dtx}
\DocInput{src/interactives/javascript.dtx}
\DocInput{src/interactives/sagemath.dtx}
\subsection{Answerables}
\DocInput{src/answer.dtx}
\DocInput{src/choice.dtx}
\DocInput{src/freeresponse.dtx}
\DocInput{src/feedback.dtx}
\DocInput{src/ungraded.dtx}
\subsection{Support for the web}
\DocInput{src/jax.dtx}
\DocInput{src/html.dtx}
\subsection{Tools}
\DocInput{src/suppress.dtx}
\DocInput{src/ending.dtx}
\section{xourse.cls}
\DocInput{src/optionsxourse.dtx}
\DocInput{src/activity.dtx}
\DocInput{src/sectioning.dtx}
\DocInput{src/graded.dtx}
\DocInput{src/logo.dtx}
\section{pgfsys-ximera.def}
\DocInput{src/pgfsys.dtx}
\end{document}
%</driver>
% \fi
%
%\def\fileversion{v2.0}
%\def\filedate{2020/01/04}
% \DoNotIndex{\newcommand,\newenvironment}
% 
%\title{\textsf{ximera} --- Simultaneously write print and online interactive materials.\thanks{This file
%   describes version \fileversion, last revised \filedate.}
%}
%\author{Jim Fowler \and Jeramiah Hocutt \and Oscar Levin \and Jason Nowell \and Hans Parshall \and Bart Snapp}
%\date{Released \filedate}
%
%\maketitle
%
%\changes{v1.0.0}{2018/05/25}{First public release}
%\changes{v2.0.0}{2020/01/04}{Second public release}
%
% \begin{abstract}
%   ``\textsf{Ximera} begins where \TeX\ ends.''  The \textsf{ximera}
%   class aids in the creation of handouts, worksheets, exercises, and
%   sections of textbooks to be used either individually or ``glued''
%   together via a \textsf{xourse} file. All \textsf{ximera} documents
%   can be deployed in an online interactive form.
%   See: \href{https://ximera.osu.edu}{Ximera Project} and the source
%   code on \href{https://github.com/XimeraProject}{GitHub}.
% \end{abstract}
%
% \section{Introduction}
%


%%%%%%%%%%%%%%%%%%%%%%%%%%%%%%%%%%%%%%%%%%%%%%%%%%%%%%%%%%%%%%%%%%
%%%%%%%%%%%%%%%%%%%%%%%%%%%%%%%%%%%%%%%%%%%%%%%%%%%%%%%%%%%%%%%%%%
%%%%%%%%%%%%%%%%%%%%%%%%%%%%%%%%%%%%%%%%%%%%%%%%%%%%%%%%%%%%%%%%%%
%%%%%%%%%%%%%%%%%%%%%%%%%%%%%%%%%%%%%%%%%%%%%%%%%%%%%%%%%%%%%%%%%%
%%%%%%%%%%%%%%%%%%%%%%%%%%%%%%%%%%%%%%%%%%%%%%%%%%%%%%%%%%%%%%%%%%


%\Finale
