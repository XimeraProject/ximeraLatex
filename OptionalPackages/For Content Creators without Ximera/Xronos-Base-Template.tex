\documentclass{article}

%%%%%%%%%%%%%%%%%%%%%
% Required Packages %
%%%%%%%%%%%%%%%%%%%%%

    \usepackage{amsmath}
    \usepackage{forloop}

%%%%%%%%%%%%%%%%%%%%
% New Environments %
%%%%%%%%%%%%%%%%%%%%

    \newenvironment{problem}{
        \textbf{\ifFirstStep Problem \else Step \arabic{ProbNum}\fi }
        \stepcounter{ProbNum}
        }
        {
        }
    \newenvironment{multipleChoice}{\textbf{Multiple Choice:} \\ \begin{enumerate}}{\end{enumerate}}

%%%%%%%%%%%%%%%%%%%%
% New Conditionals %
%%%%%%%%%%%%%%%%%%%%

    \newif\ifFirstStep
    \FirstSteptrue
    \newif\ifHW
    \HWtrue

%%%%%%%%%%%%%%%%
% New Counters %
%%%%%%%%%%%%%%%%

    \newcounter{QNumb}
    \setcounter{QNumb}{0}
    \newcounter{trackAns}
    \setcounter{trackAns}{0}
    \newcounter{IterationPH}
    \newcounter{ProbNum}

%%%%%%%%%%%%%%%%
% New Commands %
%%%%%%%%%%%%%%%%
    \newcommand{\answer}[1]{
        \overline{\underline{\left\vert \text{ \textbf{Answer:} }#1 \text{ }\right\vert}}
        }
    \newcommand{\choice}[2][incorrect]{\item #2 (You have marked this as #1}

    \newcommand{\newstep}[2]{% This identifies a new "step" in the assessment/build process.
% Syntax / Usage:
% \newstep{ Question Code }{ Answer Code }

\ifFirstStep
\else
    \begin{problem}% Begin a new "problem", unless it's the first step.
    %This is really just a step in the real problem, but code is quirky currently so it has to be listed as a "problem" for now.
\fi

#1% Display question code as you go.

#2% Followup with the answer code; so that it stays above, and not below, followup questions.

\stepcounter{trackAns}% keep track of which answer this is.

\FirstStepfalse
}
%%%

    \newcommand{\assessmentEnd}{%
\forloop{IterationPH}
	{0}%
	{\value{IterationPH} < \arabic{trackAns}}% iterate through the assessment depth one step at a time.
	{
	\end{problem}
	}
}
%%%

    \newcommand{\buildAssess}[3]{%
% Context / template
%\buildAssess{
%        Master Question/Problem Code
%    }{
%        Steps to build to the answer
%    }{
%        Master Problem Answer
%    }
\FirstSteptrue
\stepcounter{QNumb}
\begin{problem}
#1
\ifHW
#2
\begin{problem}
#3
\end{problem}
\assessmentEnd
\else% If we are making an assessment, then don't display anything but the original question.
\vfill
\fi
}
%%%

\begin{document}

\buildAssess{% First Entry is the question
    Calculate the following improper integral:
    \[
    \int -12xe^{3x} dx. 
    \]
}% End of \buildassess first argument
{% Second Argument are the steps. Each step should be entered using \newstep{ Question Code }{ Answer Code }
    \newstep{
        First, need to determine which integration technique would be appropriate here. 
    }
    {
        \begin{multipleChoice}
        \choice{We should just integrate directly, integrating the monomial term $x$ and the exponentional term $-12e^{3x}$ separately to get our integral.}
        \choice{We should integrate using (only) a $u$-substitution on the power of the exponent: $u = 3x$, and rewriting the $-12e^{3x}$ in terms of $u$ }
        \choice{We should integrate using (only) a $u$-substitution of: $u = x$, and rewriting the exponent of $-12e^{3x}$ in terms of $u$}
        \choice[correct]{We should use Integration by parts, using $u = x$ and $dv = -12e^{3x}$}
        \choice{We will should Integration by parts, using $u = -12e^{3x}$ and $dv = x$}
        \end{multipleChoice}
    }

    \newstep{
        Now that we have determined we need to use integration by parts (and chosen $u$ and $dv$), we need to compute $du$ and $v$;
    }
    {
        \[
        \begin{array}{lr}
        u = x & dv = -12e^{3x} \\
        du = \answer{1} & v = \answer{-4e^{3x}} 
        \end{array}
        \]
    }

    \newstep{
        Next we can determine the result of the integration by parts step.
    }
    {
        \[
        \int -12xe^{3x} = \answer{-4xe^{3x}} - \int \answer{-4e^{3x}}dx
        \]
    }
}% End of \buildassess second argument
{% Start of \buildassess Third Argument; the End result Question/Answer Code
    Now, using the previous steps, we can calculate our answer!
        \[
        \int -12xe^{3x} dx = \answer{-\dfrac{4}{3}(3x-1)e^{3x} + C}
        \]
}% End of \buildassess third argument


\end{document}