\begin{sagesilent}
#####Define default Sage variables.
#Default function variables
var('x,y,z,X,Y,Z')
#Default function names
var('f,g,h,dx,dy,dz,dh,df')
#Default Wild cards
w0 = SR.wild(0)

def DispSign(b):
    """ Returns the string of the 'signed' version of `b`, e.g. 3 -> "+3", -3 -> "-3", 0 -> "".
    """
    if b == 0:
        return ""
    elif b > 0:
        return "+" + str(b)
    elif b < 0:
        return str(b)
    else:
        # If we're here, then something has gone wrong.
        raise ValueError

def ISP(b):
    return DispSign(b)

def NoEval(f, c):
    # TODO
    """ Returns a non-evaluted version of the result f(c).
    """
    cStr = str(c)
    # fLatex = latex(f)
    fString = latex(f)
    fStrList = list(fString)
    length = len(fStrList)
    fStrList2 = range(length)
    for i in range(0, length):
        if fStrList[i] == "x":
            fStrList2[i] = "("+cstr+")"
        else:
            fStrList2[i] = fStrList[i]
    f2 = join(fStrList2,"")
    return LatexExpr(f2)

def HyperSimp(f):
    """ Returns the expression `f` without hyperbolic expressions.
    """
    subsDict = {
        sinh(w0) : (exp(w0) - exp(-w0))/2,
        cosh(w0) : (exp(w0) + exp(-w0))/2,
        tanh(w0) : (exp(w0) - exp(-w0))/(exp(w0) + exp(-w0)),
        sech(w0) : 2/(exp(w0) + exp(-w0)),                      # This seems to work, but Nowell said it didn't at one point.
        csch(w0) : 2/(exp(w0) - exp(-w0)),                      # This seems to work, but Nowell said it didn't at one point.
        coth(w0) : (exp(w0) + exp(-w0))/(exp(w0) - exp(-w0)),   # This seems to work, but Nowell said it didn't at one point.
        arcsinh(w0) :       ln( w0 + sqrt((w0)^2 + 1) ),
        arccosh(w0) :       ln( w0 + sqrt((w0)^2 - 1) ),
        arctanh(w0) : 1/2 * ln( (1 + w0) / (1 - w0) ),
        arccsch(w0) :       ln( (1 + sqrt((w0)^2 + 1))/w0 ),
        arcsech(w0) :       ln( (1 + sqrt(1 - (w0)^2))/w0 ),
        arccoth(w0) : 1/2 * ln( (1 + w0) / (w0 - 1) )
    }
    g = f.substitute(subsDict)
    return simplify(g)

def RandInt(a,b):
    """ Returns a random integer in [`a`,`b`]. Note that `a` and `b` should be integers themselves to avoid unexpected behavior.
    """
    return choice(range(a,b+1))

def NonZeroInt(b,c, avoid = [0]):
    """ Returns a random integer in [`b`,`c`] which is not in `av`. 
        If `av` is not specified, defaults to a non-zero integer.
    """
    while True:
        a = RandInt(b,c)
        if a not in avoid:
            return a

def RandAng(s,t):
    """ Returns a random nice angle in [0,2*pi] which lies in [`s`,`t`].
    """
    angles = [i*pi/6 for i in range(0,13)] + [(2*i+1)*pi/4 for i in range(0,4)]

    while True:
        angle = choice(angles) # Picks a random element.
        if s <= angle <= t:
            return simplify(angle)

def RandPoly(d, zeroes=[], avoid=[], monic = False, expanded = False):
    """ Returns a random polynomial of degree `d` with all the zeroes from
        `zeroes` and none of them from `avoid`.

        Note: If you want a polynomial of degree 4 with a zero at x=3 of
        multiplicity 2, for example, you should call
        >>> RandPoly(4, zeroes = [3,3], avoid = [3])
        (x-3)^2*(x+1)*(x-5)
    """
    missingDegree = d - len(zeroes)
    assert missingDegree >= 0, "Can't have degree {} with all those zeroes! {}".format(d,zeroes)

    # First, compute the part that has the specified zeroes.
    if monic:
        P = product((x-a) for a in zeroes)
    else:
        P = product((a.denominator()*x - a.numerator()) for a in zeroes)

    # Second, compute the missing factor.
    if avoid:
        missingZeroes =[NonZeroInt(-10,10,avoid) for _ in range(missingDegree)]
        missingFactor = product((x-a) for a in missingZeroes)
    else:
        if monic:
            missingFactor = x**missingDegree
        else:
            missingFactor = NonZeroInt(-10,10)*x**missingDegree

        for i in range(0,missingDegree):
            missingFactor += RandInt(-10,10)*x**i
        
    P *= missingFactor
    if expanded:
        P = expand(P)
    return P

def RandRational(c,d):
    """ Returns a random rational function with numerator and denominator of degrees `c` and `d` resp.
    """
    num_zeros = [RandInt(-10,10) for _ in range(c)] # Zeroes of the numerator
    num = RandPoly(c,zeroes=num_zeroes) # Numerator
    den = RandPoly(d,avoid =num_zeroes) # Denominator
    return num/den

def saneCheck(a,b = 1000):
    """ A function to check the reasonable-ness of an answer. It checks the 
        numerator and denominator to see if they are reasonable numbers. `a` is 
        the input number to check, and b is the threshold you want the numerator
        and denominator (if present) to be below. Default is 1000. Thus 1/596
        would pass, 5419/209 would not.
    """
    aTemp = abs(a/(7919*7907*6883)) # I need to force a denominator. Odds are, whatever you're checking won't have all these primes as factors. If they do, then you have fucked up royally anyway. Also take absolute value in order to test magnitude of values.
    
    # Initialize answer to -1 to detect if anything happened.
    check = -1
    
    if aTemp.numerator() > b or aTemp.denominator() > (b*7919*7907*6883):
        check = 0
    else:
        check = 1
    
    return check

def RandVector(b, c, avoid=[], rep=1):
    """ Returns essentially a multiset permutation of ([b,c]-av) * rep.
        That is, a vector which contains each integer in [`b`,`c`] which is not in `av` a total of `rep` number of times.

        Example:
        sage: RandVector(1,3, [2], 2)
        [3, 1, 1, 3]
    """
    oneVec = [val for val in range(b,c+1) if val not in avoid]
    vec = oneVec * rep
    shuffle(vec)
    return vec

def RandMatrix(dim1, dim2, LBnum,UBnum,LBden,UBden):
    """ Returns a random matrix with rational entries. 
        The final four parameters specify lower and upper bounds for the 
        numerators and denominators of the entries.
    """
    temp = matrix(QQ, dim1, dim2, 0)
    for i in range(dim1):
        for j in range(dim2):
            num = RandInt(LBnum, UBnum)
            den = NonZeroInt(LBden, UBden)
            temp[i,j] = num/den
    return temp

# def pad(a,n,left=True):
#   """ Returns the integer `a` padded with zeroes to be of length `n`.
#       If `left` == True, then the zeroes go on the left.
#   """
#   a = str(a)
#   missing = n - len(a) 
#   assert missing >= 0, "{} is longer than length {}!".format(a,n)
#   if left:
#       return ("0" * missing) + a
#   else:
#       return a + ("0" * missing)

\end{sagesilent}