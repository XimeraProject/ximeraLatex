\begin{sagesilent}

######  Define a function to convert a sage number into a saved counter number.

#####Define default Sage variables.
#Default function variables
var('x,y,z,X,Y,Z')
#Default function names
var('f,g,h,dx,dy,dz,dh,df')
#Default Wild cards
w0 = SR.wild(0)

def higherRoot(rootVal,rootArg):
    # Note that this returns a String version of the latex higher root in root form, rather than exponential form.
    rootString = LatexExpr(r'\sqrt[' + rootVal.str() + ']{' + rootArg.str() +'}')
    return rootString


def DispSign(b):
    """ Returns the string of the 'signed' version of `b`, e.g. 3 -> "+3", -3 -> "-3", 0 -> "".
    """
    if b == 0:
        return ""
    elif b > 0:
        return "+" + str(b)
    elif b < 0:
        return str(b)
    else:
        # If we're here, then something has gone wrong.
        raise ValueError

def ISP(b):
    return DispSign(b)

def NoEval(f, c):
    # TODO
    """ Returns a non-evaluted version of the result f(c).
    """
    cStr = str(c)
    # fLatex = latex(f)
    fString = latex(f)
    fStrList = list(fString)
    length = len(fStrList)
    fStrList2 = range(length)
    for i in range(0, length):
        if fStrList[i] == "x":
            fStrList2[i] = "("+cstr+")"
        else:
            fStrList2[i] = fStrList[i]
    f2 = join(fStrList2,"")
    return LatexExpr(f2)

def HyperSimp(f):
    """ Returns the expression `f` without hyperbolic expressions.
    """
    subsDict = {
        sinh(w0) : (exp(w0) - exp(-w0))/2,
        cosh(w0) : (exp(w0) + exp(-w0))/2,
        tanh(w0) : (exp(w0) - exp(-w0))/(exp(w0) + exp(-w0)),
        sech(w0) : 2/(exp(w0) + exp(-w0)),                      # This seems to work, but Nowell said it didn't at one point.
        csch(w0) : 2/(exp(w0) - exp(-w0)),                      # This seems to work, but Nowell said it didn't at one point.
        coth(w0) : (exp(w0) + exp(-w0))/(exp(w0) - exp(-w0)),   # This seems to work, but Nowell said it didn't at one point.
        arcsinh(w0) :       ln( w0 + sqrt((w0)^2 + 1) ),
        arccosh(w0) :       ln( w0 + sqrt((w0)^2 - 1) ),
        arctanh(w0) : 1/2 * ln( (1 + w0) / (1 - w0) ),
        arccsch(w0) :       ln( (1 + sqrt((w0)^2 + 1))/w0 ),
        arcsech(w0) :       ln( (1 + sqrt(1 - (w0)^2))/w0 ),
        arccoth(w0) : 1/2 * ln( (1 + w0) / (w0 - 1) )
    }
    g = f.substitute(subsDict)
    return simplify(g)

def RandInt(a,b):
    """ Returns a random integer in [`a`,`b`]. Note that `a` and `b` should be integers themselves to avoid unexpected behavior.
    """
    return QQ(randint(int(a),int(b)))
    # return choice(range(a,b+1))

def NonZeroInt(b,c, avoid = [0]):
    """ Returns a random integer in [`b`,`c`] which is not in `av`. 
        If `av` is not specified, defaults to a non-zero integer.
    """
    while True:
        a = RandInt(b,c)
        if a not in avoid:
            return a

def RandVector(b, c, avoid=[], rep=1):
    """ Returns essentially a multiset permutation of ([b,c]-av) * rep.
        That is, a vector which contains each integer in [`b`,`c`] which is not in `av` a total of `rep` number of times.
        Example:
        sage: RandVector(1,3, [2], 2)
        [3, 1, 1, 3]
    """
    oneVec = [val for val in range(b,c+1) if val not in avoid]
    vec = oneVec * rep
    shuffle(vec)
    return vec


def fudge(b):
    up = b+RandInt(2,5)/10
    down = b-RandInt(2,5)/10
    fudgebup = round(up,1)
    fudgebdown = round(down,1)
    fudgedb = [fudgebdown,fudgebup]
    return fudgedb

def disjointCheck(checkvec):
    if len(uniq(checkvec)) < len(checkvec):
        return 1
    else:
        return 0

def disjointIntervals(IntStart,IntEnd,CheckVal):
    if IntStart < CheckVal and CheckVal < IntEnd:
        return 1
    else:
        return 0

def IntervalVecCheck(checkVec):
    veclen = len(checkVec)
    returnval = 0
    for i in range(veclen):
        for j in range(veclen):
            if (disjointIntervals(checkVec[j][0],checkVec[j][1],checkVec[i][0]) + disjointIntervals(checkVec[j][0],checkVec[j][1],checkVec[i][1])) > 0:
                returnval = returnval + 1
    if returnval > 0:
        return 1
    else:
        return 0




\end{sagesilent}