\documentclass{ximera}
\title{FAQ, or  Frequently Asked Questions}
\begin{document}

\begin{abstract}
Below are a few questions you might have about Ximera or about the
format of the courses hosted here.  If you don't find your question
answered here, please contact us.
\end{abstract}

\maketitle

\section{What is Ximera?}

Ximera is the interactive textbook platform created through NSF Grant
DUE--1245433.

%NSF Grant
%DUE-1245433](http://www.nsf.gov/awardsearch/showAward?AWD_ID=1245433).

\section{What does it do?}

A content author uploads classroom activities to GitHub; the Ximera
backend server downloads these activities and presents them to
students.

\section{Who is Ximera for?}

Whether you are an instructor or a student, Ximera has something to offer you.

\section{Why is it called Ximera?}

The X is a capital chi, so Ximera is pronounced like chimera, and a
chimera is a mythical beast composed of three different animals.
Likewise, Ximera glues together LaTeX, JavaScript, and HTML to produce
a compelling student experience.

\section{I am interested in producing a course for Ximera; how can I make my own course?}

Please contact us.  We'd love to talk about how we can work together.

\section{I would like to contribute to the backend technology.  How do I help?}

The backend code that Ximera provides is all available on
GitHub.

\section{I am a teacher.  Can I provide material from Ximera to my students?}

Yes, absolutely!  Please feel free to use the material you find here
for your own students.  We'd love to hear about how we can make the
material more useful to you.

\section{When do the courses start and end?}

The courses on Ximera are self-paced; you can join in at any time.

\end{document}
